\chapter{Introdução}
\label{cap_introducao}
A utilização de combustíveis fósseis, o aquecimento global e o aumento da população mundial, que reflete-se no aumento na demanda por energia, têm se tornado tópicos cada vez mais importantes no cenário global. Estes fatos têm acelerado o desenvolvimento de fontes renováveis de energia, como a eólica ou solar, que prometem causar menos impactos ambientais. A integração dessas novas fontes de energia aos sistemas existentes de distribuição e transmissão, demanda a existência de um novo \emph{grid} de comunicação que ofereça uma infra-estrutura mais rápida, eficiente e confiável \cite{Art-Gungor2013}.

Ainda segundo \cite{Art-Gungor2013}, as tecnologias de energia renovável possuem certa tendência natural de instalação não centralizada. Exemplos disso são, os geradores que se utilizam de células fotovoltaicas ou os geradores eólicos, já que ambos aproveitam-se de características geográficas e/ou climáticas. Ao considerar o sistema de distribuição de energia atual é fácil perceber que ele foi construído para o escoamento de energia proveniente de um sistema de geração altamente centralizado. Obviamente isto não significa que exista apenas um ponto de geração de energia, mas sim que eles são pontuais. Logo, a inserção destas tecnologias geradoras de energia renovável deve transformar o sistema de distribuição de energia atual em um grande sistema de geração distribuída de energia, tornando-se dessa forma, imperativo a utilização de uma infraestrutura capaz de suprir as necessidades de comunicação demandadas pela variedade de sensores, atuadores e elementos utilizados no controle desta nova rede de geração e distribuição de energia. O que, em resumo, representa o próprio conceito de \emph{Smart Grid} (\sigla{SG}{\emph{Smart Grid}}).

Este conceito de infraestrutura refere-se a uma completa modernização no sistema de energia visando monitorar, proteger e otimizar a operação dos elementos utilizados na rede. Englobam-se aqui os elementos desde os mais básicos como os presentes em residências e utilizados por consumidores finais, até os mais críticos como os elementos geradores/armazenadores de energia ou aqueles presentes nas redes de alta tensão \cite{Conf-Sood2009}.

Este novo cenário de Geração Distribuída de energia (\sigla{DG}{Distributed Generation} - Distributed Generation) viabiliza a aplicação de novos métodos na análise e resolução de problemas na rede de distribuição e na segurança/estabilidade da mesma, como a utilização de \emph{Self Healing} por exemplo \cite{Art-Amin2006}. Mas para isso, além da existência de uma rede de comunicação inteligente ou SG, também faz-se necessário o atendimento de requisitos específicos de comunicação \cite{Conf-Sood2009}.

O SG necessita de grande flexibilidade para permitir a adição de serviços cada vez mais variados e com diferentes requisitos de comunicação, como monitoramento de sensores em tempo real e interação com atuadores e usuários finais \cite{Art-Aggarwal2010}. Sabe-se que uma rede \sigla{TCP/IP}{\emph{Transmission Control Protocol/Internet Protocol}} deve atender, com elevado grau de satisfação, os requisitos de comunicação demandados \cite{Conf-Lobo2008}. Requisitos estes que, segundo \cite{Art-Aggarwal2010}, devem ser substancialmente consideráveis tanto em largura de banda quanto em valores de latência, justificando facilmente o uso de um meio físico como a fibra óptica assim como qualquer tentativa de redução de latência. 
%a exemplo da proposta contida neste documento.

O presente trabalho apresenta uma pesquisa sobre os protocolos de roteamento mais utilizados em redes SG, enfatizando o viés da redução de latência. Além disso, são apresentados resultados de simulações quanto a utilização de chaveadores ópticos em pontos estratégicos da rede (considerando uma infraestrutura de rede construída com fibras ópticas) buscando \emph{by-passar} equipamentos sem prejudicar nenhuma comunicação, reduzindo a latência de rede.

\section{Motivação}
As redes SG devem propiciar a utilização/surgimento de diversos novos serviços que, potencialmente, servirão como porta de entrada de novas tecnologias no dia-a-dia da sociedade. Dessa forma pode-se imaginar que os requisitos demandados das redes de comunicação, base das SG, sejam cada vez mais exigentes. Neste sentido qualquer esforço realizado para otimização dessas redes, e consequente diminuição de latência ou aumento de largura de banda, são bem-vindos.

Através da utilização de chaveadores ópticos estrategicamente posicionados em uma rede, potencialmente mas não necessariamente \sigla{IP}{\emph{Internet Protocol}}, pode-se maximizar a performance da mesma em termos de latência. Os chaveadores ópticos, se empregados conforme proposto no presente documento, podem diminuir o número de equipamentos pelos quais os dados devem passar até chegar ao seu destino, otimizando a rede em questão. Para isto, faz-se necessária a criação de um protocolo específico para ativação dos mesmos.

\section{Objetivos}
\subsection{Objetivo Geral}
Realizar um estudo sobre os protocolos de roteamento usados em redes SG e propor uma metodologia de roteamento/protocolo eficiente em redução de latência através da alteração física da topologia de rede advinda da utilização de chaves ópticas estrategicamente posicionadas e acionadas.
\subsection{Objetivos Específicos - CONFIRMAR}
\begin{itemize}
	\item Realizar um estudo teórico sobre os protocolos de roteamento existentes utilizados em redes SG.
	\item Comparar a performance dos protocolos estudados para a aplicação em questão.
	\item Elaborar e apresentar uma abordagem e proposta de roteamento baseado na alteração da estrutura física da rede através da utilização de chaveadores ópticos e apresentar dados de simulação.
\end{itemize}