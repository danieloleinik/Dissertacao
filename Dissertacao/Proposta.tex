\chapter{Proposta}
Este capítulo destina-se à descrição da proposta do presente documento.

Conforme descrito na seção \ref{cap_protocolos_de_roteamento} um protocolo de roteamento é responsável pelo tratamento e pela dinâmica das mensagens e dados trocados entre os dispositivos participantes da comunicação. Como dinâmica pode-se citar o cálculo de rotas a serem utilizadas e, neste sentido, a proposta do presente documento pode ser considerada como um protocolo para estabelecimento de caminhos ótimos de rede através da manipulação do \emph{layer} físico da rede.

objetivo é a descoberta e estabelecimento de rotas completamente desagregado do protocolo de roteamento utilizado na rede em questão. O protocolo proposto pode ser utilizado em qualquer rede que contenha multi-caminhos ou redundâncias, apesar de apresentar melhores resultados quando aplicado à redes \emph{mesh} devido a maior possibilidade de recombinações. Para tanto, é necessário que a rede em questão possa ser representada por um grafo simples, ou seja, um grafo que não possui arestas paralelas ou pontas coincidentes (laços).

De maneira geral, uma rede \emph{mesh} apresenta vários possíveis caminhos entre o ponto inicial e destino da comunicação, sendo que dessa forma existem custos atrelados ao roteamento, encaminhamento e processamento de pacotes em qualquer comunicação neste tipo de topologia. A proposta atua no refino destes custos através do estabelecimento de caminhos específicos/estratégicos para cada enlace.

Basicamente existem dois tipos de equipamentos que podem ser utilizados em redes ópticas, os puramente ópticos e os elétricos. Os equipamentos puramente ópticos possuem custo muito elevado e são muito sensíveis, já que têm a capacidade de realizar o roteamento de luz (baseado no comprimento de onda dos sinais) e por isso não são utilizados em larga escala, normalmente esse tipo de tecnologia pode ser encontrado em alguns filtros e \emph{splitters} especiais. Já os equipamentos elétricos possuem custos muito mais atrativos e são normalmente empregados para roteamento de maneira geral, mas em contrapartida são sensivelmente mais lentos.

Os atrasos existentes em equipamentos elétricos utilizados em redes ópticas têm diversas origens, como o processamento dos dados (intrinsecamente relacionado ao poder de processamento do equipamento) e as conversões de meio necessárias, sendo que para cada equipamento pode-se considerar pelo menos duas dessas conversões (óptica-elétrica-óptica). 

Logo, todas as vezes que um pacote trafega em uma rede óptica, acaba sofrendo atrasos que são traduzidos em forma de latência, que devem ser somadas a fim de determinar a latência total de um enlace. A Figura \ref{fig_latency_link} mostra a acúmulo da latência através de um enlace óptico entre dois pontos. 

\begin{figure} [!htb]% normalmente utilizar [!t]
	\centering
	\includegraphics[width=0.5\textwidth]{./figuras/latency-link.png}
	\caption[Latência de Link]{Latência em comunicação entre dois pontos utilizando link óptico. Os principais pontos de inserção de latência na comunicação são resultantes dos processos de conversão elétrica-óptica, transmissão de dados, conversão óptica-elétrica e disponibilização dos dados no destino (desserialização e processamento).}
	\fonte{Adaptado de \cite{Art-Coffe2017}}
	\label{fig_latency_link}
\end{figure}

Logo, a proposta apresentada neste artigo, denominada de \emph{``Bypass Seletivo de Nós''} (BSN), ocupa um lugar intermediário entre os equipamentos puramente ópticos e os ópticos/elétricos. O BSN consiste em um protocolo especificamente desenvolvido para otimização de latência em redes ópticas através da utilização de chaveadores ópticos em pontos estratégicos da rede. Dessa forma equipamentos em posições específicas da rede podem ser desviados sem a necessidade de nenhuma conversão de meio ou processamento de sinais, resultando na mínima inserção de latência possível. 

O método de redução do grafo de rede utilizado pelo BSN é descrito sucintamente em alto nível através do pseudocódigo representado no Algoritmo \ref{pseudocode_bsn}. O BSN utiliza vários conceitos de otimização simultaneamente. A ideia básica é a utilização de um algoritmo genético para a criação de diferentes combinações de estado dos chaveadores óticos na rede (de acordo com as premissas de projeto a serem adotadas). Dessa forma, cada uma das combinações geradas pelo algoritmo genético acaba por gerar uma nova topologia de conexões físicas, que são por sua vez otimizadas buscando obter a rede menos profunda possível. Para o passo representado na linha \ref{bsn:fitness_line} do Algoritmo \ref{pseudocode_bsn} (busca local), pode ser utilizado qualquer algoritmo conhecido sendo que o presente documento utilizou uma busca de custo uniforme, garantindo assim a obtenção do valor ótimo de \emph{fitness}.

\begin{algorithm} [h]
\caption{ - Algoritmo básico do BSN}
\begin{algorithmic}[1]
\State $popsize\gets \textit{tamanho da populacao}$
\State $gennumb\gets \textit{numero de geracoes}$\\
\State PopList[] = new List[popsize]\Comment{Variável para alocação da geração atual}
\State $i\gets 0$
\While {$ i< popsize $}
\State Individuo = CriaIndividuo()\Comment{Cria a uma configuração de rede}
\State CalculaFit(Individuo)\label{bsn:fitness_line}\Comment{Realiza busca local na instância de rede}
\State PopList.Add(Individuo)\Comment{Adiciona a instância de rede à lista}
\State PopList.ClassifcaIndividuos()\Comment{Organiza a população atual de acordo com o Fitness individual}
\EndWhile\Comment{População inicial criada}\\
\State $j\gets 1$
\While {$ j< gennumb $}\Comment{Roda algoritmo genético}
\State ParentsList[] = new List[]\Comment{Variável para alocação de indivíduos ``pais''}
\State NewList[] = new List[popsize]\Comment{Variável para alocação de próxima geração}
\State ParentsList = SelecionaIndividuos(PopList)\Comment{Seleciona os indivíduos da geração atual que serão utilizados para criação da geração seguinte}
\State NewList = CrossOver(ParentsList)\Comment{Cruza os indivíduos selecionados}
\State Mutation(NewList)\Comment{Aplica algoritmo de mutação de genes na nova população}
\State PopList = MesclaGen(NewList, PopList)\Comment{Realiza elitismo de indivíduos}
\State PopList.ClassifcaIndividuos()\Comment{Organiza a população atual de acordo com o Fitness individual}
\EndWhile\\
\State\Return PopList.Individuo(0)\Comment{Retorna melhor instância de rede}
\end{algorithmic}
\label{pseudocode_bsn}
\end{algorithm}

Os pontos da rede de acionamento do chaveador óptico podem ser escolhidos em ramos de um grafo com muitos vértices (\emph{hops}) ou podem ser planejados para o estabelecimento de rotas específicas para comunicação com pontos críticos do sistema, estabelecendo um \sigla{SLA}{\emph{Service Level Agreement}} em \emph{layer} físico.

Através da utilização do BSN é possível reduzir o grafo da rede possibilitando uma nova configuração de conexões e consequentemente uma nova variedade de rotas possíveis. Cada vez que o BSN ativa um \emph{bypass}, as duas arestas envolvidas são unificadas evitando o vértice das mesmas, criando assim um novo grafo com uma distância entre os dois vértices finais de um vértice a menos, como representado na Figura \ref{fig-bypass-exemplo}. Dessa forma, em caso de falha ou mudança de SLA, a topologia física pode ser alterada automaticamente.

\begin{figure} % normalmente utilizar [!t]
	\centering
	\includegraphics[width=0.5\textwidth]{./figuras/Bypass-exemplo-PB.png}
	\caption[Exemplo de atuação de \emph{by-pass} óptico]{Efeito da utilização do chaveador ótico. Em ``a'' pode-se ver a rede original contendo dois saltos entre os vértices A e C. Em ``b'' após o acionamento do dispositivo no vértice B pode-se ver e existência de apenas um salto entre os vértices A e C}
	\label{fig-bypass-exemplo}
\end{figure}