\chapter{Conclusão}

Foi exposto um trabalho de pesquisa com relação a redes de computadores e a utilização de componentes capazes de provocar alterações em camada física, como os chaveadores ópticos. Foram mostrados conceitos básicos relacionados a protocolos de roteamento assim como alguns protocolos relacionados ao ambiente ao qual a proposta se destina. Também estudaram-se os principais requisitos e serviços de redes inteligentes de distribuição de energia elétrica, verificando-se que alguns desses serviços possuem requisitos severos em termos de latência, principalmente em se tratando de serviços de proteção e automação.

Além disso foram mostrados conceitos básicos de computação evolucional, empregando-se o conceito de algoritmos genéticos na criação da proposta de um protocolo, nomeado de BSN - ``\emph{Bypass} Seletivo de Nós'', destinado a melhorar os valores de latência de uma rede óptica através da manipulação de chaveadores ópticos estrategicamente selecionados, de maneira a diminuir o número de saltos entre a origem e destino de uma comunicação.

Para avaliação de desempenho do protocolo proposto foi desenvolvido um programa de simulação capaz de gerar grafos aleatórios seguindo uma distribuição de probabilidade de interconexão condizente com a realidade do ambiente ao qual a proposta se destina. Os grafos resultantes foram então submetidos ao algoritmo do BSN gerando os resultados utilizados.

De acordo com o apresentado na seção \ref{resultados}, fica claro que a efetividade do algoritmo proposto está intrinsecamente ligada à topologia da rede que será submetida ao mesmo. O que já era esperado, visto que a possibilidade de rearranjo da camada física conseguida através da utilização dos chaveadores ópticos e consequente alteração da topologia resultante depende de como e quantos nós estão interligados na rede original. Mesmo assim, pode-se notar uma sensível melhora nos parâmetros de custo ou profundidade da rede resultante em detrimento de sua largura quando comparada ao grafo da rede original. Em outras palavras, o BSN tem a capacidade de converter uma rede profunda em uma rede larga, obviamente sujeito à limitações de conexão já existentes na rede original.

Também pode-se notar que, ao seguir uma distribuição de probabilidades de interconexão de nós condizente com a realidade a qual a proposta se destina como a representada na Figura \ref{fig_dist_prob}, a valência máxima utilizada na geração do grafo não tem grande impacto no percentual de melhoria alcançado através da utilização do BSN. Em contrapartida, fica claro uma diminuição de efetividade do protocolo na melhoria do custo total da rede quando usado em redes com muitos nós. Aqui é importante ressaltar que o custo total, definido conforme o exposto em \ref{subsection-custo-da-rede}, é na verdade a soma de todos os custos entre o nó central e todos os destinos da rede. Neste sentido, conclui-se que existem pelo menos dois fatores contribuintes para a diminuição do percentual de melhoria com o aumento do número de nós da rede. O primeiro é inerente ao fato de que espera-se apenas algumas conexões com grandes valores de custo, ou seja, a rede como um todo deve possuir muito mais conexões com baixo custo do que com alto custo, conforme mostrado na Figura \ref{fig_graph_ramos_custo}, sendo que desta forma a contribuição da otimização destes ramos para o custo total da rede acaba sendo pequena. O segundo fator contribuinte é o de que quanto mais nós existirem em uma rede \emph{mesh}, maior será a possibilidade de combinações/rotas disponíveis. Desta forma, em uma rede com alto nível de multipercursos espera-se que as rotas ótimas já possuam baixo custo sem a necessidade da alteração de topologia.

Logo, pode-se concluir que, com relação ao custo total da rede, quanto maior o diâmetro da mesma menor será a efetividade do algoritmo quando considerando a rede como um todo. Em contrapartida o mesmo princípio não se aplica ao considerar-se o custo individual de cada um dos ramos do grafo, o que em geral representa a maior preocupação deste tipo de aplicação.

Conforme o apresentado em \ref{melhoria-custo-individual}, quando maior o número de saltos de um ramo em específico maior será o percentual de melhoria conseguido em seu custo através da utilização do BSN. Fato esperado visto que quanto mais comprido é um ramo do grafo mais efetiva deve ser a utilização do chaveador óptico na alteração da topologia física desta conexão. Dessa forma, além de ótimo resultado na diminuição do custo total da rede, o BSN apresenta um ótimo resultado ao considerar-se ramos longos do grafo, abrindo a oportunidade da utilização do algoritmo para otimização, não apenas da rede como um todo, mas também da comunicação com nós de baixo desempenho devido a profundidade de rede.

Por fim, é interessante ressaltar que a função objetivo ou função de \emph{fitness} utilizada nesta implementação teve foco na redução de número de saltos considerando a rede completa como objeto de avaliação, mas a mesma poderia ser alterada para considerar a minimização apenas dos ramos mais compridos, com pior desempenho, maior prioridade ou ainda basear-se na distribuição de carga ou indicadores como o QoS sendo estas implementações sugeridas para trabalhos futuros.

Implementações como estas podem verificadas, conforme comentado, através da alteração da função de \emph{fitness} do BSN mas é necessário ressaltar que a validação do protocolo seria dificilmente provada neste caso já que a decisão do nó ou equipamento alvo da otimização depende basicamente do administrador da rede em questão.


